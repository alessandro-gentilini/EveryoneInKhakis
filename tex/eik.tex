\documentclass[12pt,letterpaper]{article}

\newcommand{\equationname}{equation}

\begin{document}

The Jeans Equations are remarkable: They relate the second moments of
velocities (as a function of position) to the gravitational potential
and the density of the tracer particles.
And they are provably correct under any distribution function.
That is, they rely on having a well-mixed set of tracer particles,
and knowing things about the distribution of those tracers in position space.
But once those things are true---the tracers are well mixed and have a known
distribution in position space---the Jeans equation tells you how the second
moments of the velocities of those particles are related to the potential.
And that doesn't depend on the distribution function!
That sounds magical!
Can't we use the Jeans equations to turn velocity measurements into
gravitational-potential inferences and learn about the potential without
making any assumptions whatsoever about the distribution function in
actions or orbits?

Of course the Jeans Equations are \emph{equations}, and inferences require
some inferential mechanism.
How do we turn the Jeans Equations into inferences?
It turns out that we have a lot of \emph{choices} when we do that!
And those choices are material: They will be differently appropriate in
different situations.

Furthermore, since the Equations give you relationships between velocity moments
and potentials, they don't imply, in any sense, a \emph{generative model}.
That is, the Jeans Equations don't imply any kind of likelihood function or
probability for the data given parameters.
Of course they don't: Their very magic is their independence from any particular
distribution function.
That is, the magic that makes the Jeans Equations independent of distribution
function makes the Jeans equations useless for building any kind of likelihood
function or predictive distribution for the data.

That's a bunch of generalities! Let's get specific.
According to Binney \& Tremaine \equationname~(4.271) the one-dimensional vertical
Jeans Equation in a slab (normal to the $z$ direction) is:
\begin{equation}
\frac{\partial(\nu\,\overline{v_z^2})}{\partial z} + \nu\,\frac{\partial\Phi}{\partial z} = 0
\quad ,
\end{equation}
where
$\nu$ is the number density of tracers,
$\overline{v_z^2}$ is the mean of the square of the $z$-direction velocity of those tracers,
$\Phi$ is the gravitational potential,
and all quantities are a function of $z$ only.
This is a remarkable equation, and increidibly constraining!
You can use it to turn velocity measurements into potential inferences.
But how, exactly?

Well, the potential derivative looks like a derivative of a second
moment of the velocity data.
Cool.
Or the mass density (which is often the goal) looks like a second
derivative of a second moment of the velocity data.
Cool.
Or is it cool?
Most people in data science would tell you that a second moment of
your data is a lot less well constrained than most first moments.
And they would tell you that you don't really want to be taking
derivatives of your data, and certainly not second derivatives.
But let's proceed for a bit just to see what we get from this.

HOGG...

\end{document}
